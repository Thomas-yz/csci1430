%%%%%%%%%%%%%%%%%%%%%%%%%%%%%%%%%%%%%%%%%%%%%%%%%%%%%%%%%%%%%%%%%%%%%
%
% CSCI 1430 Writeup Template
%
% This is a LaTeX document. LaTeX is a markup language for producing
% documents. Your task is to fill out this
% document, then to compile this into a PDF document.
%
% TO COMPILE:
% > pdflatex thisfile.tex
%
% For references to appear correctly instead of as '??', you must run
% pdflatex twice.
%
% If you do not have LaTeX and need a LaTeX distribution:
% - Departmental machines have one installed.
% - Personal laptops (all common OS): www.latex-project.org/get/
%
% If you need help with LaTeX, please come to office hours.
% Or, there is plenty of help online:
% https://en.wikibooks.org/wiki/LaTeX
%
% Good luck!
% James and the 1430 staff
%
%%%%%%%%%%%%%%%%%%%%%%%%%%%%%%%%%%%%%%%%%%%%%%%%%%%%%%%%%%%%%%%%%%%%%
%
% How to include two graphics on the same line:
%
% \includegraphics[\width=0.49\linewidth]{yourgraphic1.png}
% \includegraphics[\width=0.49\linewidth]{yourgraphic2.png}
%
% How to include equations:
%
% \begin{equation}
% y = mx+c
% \end{equation}
%
%%%%%%%%%%%%%%%%%%%%%%%%%%%%%%%%%%%%%%%%%%%%%%%%%%%%%%%%%%%%%%%%%%%%%%%%%%%%%%%%%%%%%%%%%%%%%%%%

\documentclass[11pt]{article}

\usepackage[english]{babel}
\usepackage[utf8]{inputenc}
\usepackage[colorlinks = true,
            linkcolor = blue,
            urlcolor  = blue]{hyperref}
\usepackage[a4paper,margin=1.5in]{geometry}
\usepackage{stackengine,graphicx}
\usepackage{fancyhdr}
\setlength{\headheight}{15pt}
\usepackage{microtype}
\usepackage{times}
\usepackage{booktabs}

% python code format: https://github.com/olivierverdier/python-latex-highlighting
\usepackage{pythonhighlight}

\frenchspacing
\setlength{\parindent}{0cm} % Default is 15pt.
\setlength{\parskip}{0.3cm plus1mm minus1mm}

\pagestyle{fancy}
\fancyhf{}
\lhead{Project 1 Writeup}
\rhead{CSCI 1430}
\rfoot{\thepage}

\date{}

\title{\vspace{-1cm}Project 1 Writeup}


\begin{document}
\maketitle
\vspace{-2cm}
\thispagestyle{fancy}

\section*{Instructions}
\begin{itemize}
  \item Provide an overview about how your project functions. 
  \item Describe any interesting decisions you made to write your algorithm.
  \item Show and discuss the results of your algorithm.
  \item Feel free to include code snippets, images, and equations.
  \item List any extra credit implementation and result (optional).
  \item Use as many pages as you need, but err on the short side.
  \item \textbf{Please make this document anonymous.}
\end{itemize}

\section*{Project Overview}

This project aims to implement two functions \begin{verbatim}myfilter()\end{verbatim} and \begin{verbatim}gen_hybrid_image()\end{verbatim}. Different types of filters are applied to images to test \begin{verbatim}myfilter()\end{verbatim} function. As for \begin{verbatim}gen_hybrid_image()\end{verbatim}, besides testing with extra pairs of images, the effect of image assignment sequence to \begin{verbatim}image1\end{verbatim} and \begin{verbatim}image2\end{verbatim} is also investigated.

\section*{Implementation Detail}

Here is the implemetation for \begin{verbatim}myfilter()\end{verbatim}
\begin{python}

\end{python}
    (k, l) = kernel.shape
    (m, n, c) = image.shape
    if (k * l) % 2 == 0:
        raise Exception("Output with even filters are not defined!")

    Grayscale = False
    if len(image.shape) == 2:
        Grayscale = True
        image = np.reshape(image, (image.shape[0], image.shape[1], 1))

    padded_image = np.pad(
        image, ((k // 2, k // 2), (l // 2, l // 2), (0, 0)), "constant"
    )
    # because we want to calculate convolution, we need to flip the kernel
    flipped_kernel = np.flip(kernel)
    output = np.zeros(image.shape)
    for o in range(c):
        for i in range(m):
            for j in range(n):
                output[i, j, o] = np.tensordot(
                    flipped_kernel, padded_image[i : i + k, j : j + l, o]
                )

    if Grayscale:
        output = output.reshape(output, (m, n))
    filtered_image = output
\section*{Result}

\begin{enumerate}
    \item Result 1 was a total failure, because...
    \item Result 2 (Figure~\ref{fig:result1}, left) was surprising, because...
    \item Result 3 (Figure~\ref{fig:result1}, right) blew my socks off, because...
\end{enumerate}

\begin{figure}[h]
    \centering
    \includegraphics[width=5cm]{placeholder.jpg}
    \includegraphics[width=5cm]{placeholder.jpg}
    \caption{\emph{Left:} My result was spectacular. \emph{Right:} Curious.}
    \label{fig:result1}
\end{figure}

My results are summarized in Table~\ref{tab:table1}.

\begin{table}[h]
    \centering
    \begin{tabular}{lr}
        \toprule
        Condition & Time (seconds) \\
        \midrule
        Test 1 & 1 \\
        Test 2 & 1000 \\
        \bottomrule
    \end{tabular}
    \caption{Stunning revelation about the efficiency of my code.}
    \label{tab:table1}
\end{table}

\section*{Extra Credit (Optional)}
\begin{enumerate}
   
    \item Pad with reflected image content
        \begin{python}
        one = 1;
        two = one + one;
        if two == 2
            disp( 'This computer is not broken.' );
        end
        \end{python}
    
    \item own hybrid image
        \begin{python}
        one = 1;
        two = one + one;
        if two == 2
            disp( 'This computer is not broken.' );
        end
        \end{python} 

    \item FFT-based convolution
        \begin{python}
        one = 1;
        two = one + one;
        if two == 2
            disp( 'This computer is not broken.' );
        end
        \end{python} 
\end{enumerate}

\end{document}
