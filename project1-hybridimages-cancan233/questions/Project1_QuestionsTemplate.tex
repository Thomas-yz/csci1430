%%%%%%%%%%%%%%%%%%%%%%%%%%%%%%%%%%%%%%%%%%%%%%%%%%%%%%%%%%%%%%%%%%%%%%%%%%%%%%%%%%%%%%%%%%%%%%%%
%
% CSCI 1430 Written Questions
%
% This is a LaTeX document. LaTeX is a markup language for producing documents.
% Your task is to answer the questions by filling out this document, then to
% compile this into a PDF document.
% You will then upload this PDF to `Gradescope' - the grading system that we will use.
% Instructions for upload will follow soon.
%
%
% TO COMPILE:
% > pdflatex thisfile.tex
%
% If you do not have LaTeX and need a LaTeX distribution:
% - Departmental machines have one installed.
% - Personal laptops (all common OS): http://www.latex-project.org/get/
%
% If you need help with LaTeX, come to office hours. Or, there is plenty of help online:
% https://en.wikibooks.org/wiki/LaTeX
%
% Good luck!
% James and the 1430 staff
%
%%%%%%%%%%%%%%%%%%%%%%%%%%%%%%%%%%%%%%%%%%%%%%%%%%%%%%%%%%%%%%%%%%%%%%%%%%%%%%%%%%%%%%%%%%%%%%%%
%
% How to include two graphics on the same line:
%
% \includegraphics[width=0.49\linewidth]{yourgraphic1.png}
% \includegraphics[width=0.49\linewidth]{yourgraphic2.png}
%
% How to include equations:
%
% \begin{equation}
% y = mx+c
% \end{equation}
%
%%%%%%%%%%%%%%%%%%%%%%%%%%%%%%%%%%%%%%%%%%%%%%%%%%%%%%%%%%%%%%%%%%%%%%%%%%%%%%%%%%%%%%%%%%%%%%%%

\documentclass[11pt]{article}

\usepackage[english]{babel}
\usepackage[utf8]{inputenc}
\usepackage{amssymb}
\usepackage{xcolor}
\usepackage[colorlinks = true,
            linkcolor = blue,
            urlcolor  = blue]{hyperref}
\usepackage[a4paper,margin=1.5in]{geometry}
\usepackage{stackengine,graphicx}
\usepackage{fancyhdr}
\setlength{\headheight}{15pt}
\usepackage{microtype}
\usepackage{times}
\usepackage[shortlabels]{enumitem}
\setlist[enumerate]{topsep=0pt}
\usepackage{amsmath}
% \usepackage{subfig}
\usepackage{caption}
\usepackage{subcaption}


% a great python code format: https://github.com/olivierverdier/python-latex-highlighting
\usepackage{pythonhighlight}

\frenchspacing
\setlength{\parindent}{0cm} % Default is 15pt.
\setlength{\parskip}{0.3cm plus1mm minus1mm}

\pagestyle{fancy}
\fancyhf{}
\lhead{Project 1 Questions}
\rhead{CSCI 1430}
\rfoot{\thepage}

\date{}

\title{\vspace{-1cm}Project 1 Written Questions Solutions}


\begin{document}
\maketitle
\vspace{-3cm}
\thispagestyle{fancy}

\section*{Instructions}
\begin{itemize}
  \item 6 questions.
  \item Include code, images, and equations where appropriate.
  \item Please make this document anonymous.
  \item When you are finished, compile this document to a PDF and submit it directly to Gradescope. On upload, \textbf{Gradescope will ask you to assign question numbers to your pages}. Making each question end with a page break after your answer is a good way to ease this process.
  \item This assignment is \textbf{fixed length}, and the pages have been assigned for you in Gradescope. As a result, \textbf{please do NOT add any new pages}. We will provide ample room for you to answer the questions. If you \emph{really} wish for more space, please add a page \emph{at the end of the document}.
  \item \textbf{We do NOT expect you to fill up each page with your answer.} Some answers will only be a few sentences long, and that is okay.
\end{itemize}

\section*{Questions}

\paragraph{Q1:} Image convolution, a type of image filtering, is a fundamental image processing tool that you will use repeatedly throughout the course.

\begin{enumerate}[(a)]
\item \emph{Explicitly describe} the 3 main components of image convolution:
\begin{enumerate}[(i)]
    \item input
    \item transformation (how it happens)
    \item output
\end{enumerate}
\item Why is image convolution important in Computer Vision? Which applications does it allow?
\end{enumerate}


%%%%%%%%%%%%%%%%%%%%%%%%%%%%%%%%%%%
\pagebreak
\paragraph{A1:} Your answer here.
% Uncomment the stencil below and fill in your solution.

\begin{enumerate}[(a)]
  \item  
\begin{enumerate}[(i)]
  \item Input is the image $I$ of size m $\times$ n 
  \item Transformation is the filter $f$ of size k $\times$ l
  \item output is the same size as image but with form of $\sum_{k,l} f[k,l]I[m-k, n-l]$
\end{enumerate}
  \item Because image convolution allows us to learn which features in pictures that provide useful information in images. It also allows us to enhance images (e.g. increase contrast, fake tilt shift) and detect patterns.
\end{enumerate}

%%%%%%%%%%%%%%%%%%%%%%%%%%%%%%%%%%%

% Please leave the pagebreak
\pagebreak
\paragraph{Q2:} Correlation and convolution are both filtering operations we can perform to extract information from images.
\begin{enumerate}[(a)]
    \item 
    What is the difference between convolution and correlation?

    \item
    Construct a scenario which produces a different output between both operations. Include the kernel you used and your image results. Use your understanding of convolution and correlation to explain the outputs.
    
    \emph{Please use \href{https://docs.scipy.org/doc/scipy/reference/generated/scipy.ndimage.convolve.html}{$scipy.ndimage.convolve$} and \href{https://docs.scipy.org/doc/scipy/reference/generated/scipy.ndimage.correlate.html}{$scipy.ndimage.correlate$} to experiment!}
    
\end{enumerate}


%%%%%%%%%%%%%%%%%%%%%%%%%%%%%%%%%%%
\paragraph{A2:} Your answer here.
% Uncomment the stencil below and fill in your solution.

\begin{enumerate}[(a)]

\item The difference between convolution and correlation is the operation symbol before the filter element, where convolution is minus and correlation is plus. (See the equations for 2d correlation and 2d convolution below). The convolution filter is flipped by 180 degrees. If the filter kernel is symmetric, correlation and convolution are identical.
\begin{equation*}
    \begin{split}
        h[m, n] &= \sum_{k,l}f[k,l]I[m+k, n+l] \qquad \text{2d correlation} \\
        h[m, n] &= \sum_{k,l}f[k,l]I[m-k, n-l] \qquad \text{2d convolution}
    \end{split}
\end{equation*}

\item As long as the filter is nonsymmetric, the output between convolution and correlation is different. Here I use the '../data/fish.bmp' as an example. The input image, filter and output images are shown  below. The code to obtain the results is attahced at last page. First I read the image into grayscale for sake of simplicity. As brightness in image increases, the response in output will increase. Overall brighter regions will give higher correlation response. To avoid this, the mean is substracted for both input image and filter. \\
In the output of correlation, the rigion is bright where the filter is similar to the input image. Thus, we can see that correlation output is bright around the eye of the fish because the filter is the eye of the fish. And for the output of convolution, because the different operation of filter applied to the input image, the result is contrast to the output of correlation.

\begin{figure}[htbp]
    \centering
    \begin{subfigure}[htbp]{0.24\textwidth}
        \centering
        \includegraphics[width=\textwidth]{fish.png}
        \caption{fish input image: 307 $\times$ 375}
        \label{fish}
    \end{subfigure}
    \hfill
    \begin{subfigure}[htbp]{0.24\textwidth}
        \centering
        \includegraphics[width=\textwidth]{filter_fish.png}
        \caption{filter: 50 $\times$ 50}
        \label{filter_fish}
    \end{subfigure}
    \hfill
    \begin{subfigure}[htbp]{0.24\textwidth}
        \centering
        \includegraphics[width=\textwidth]{correlate2d_fish.png}
        \caption{correlate2d output image: 307 $\times$ 375}
        \label{correlate2d_fish}
    \end{subfigure}
    \hfill 
    \begin{subfigure}[htbp]{0.24\textwidth}
        \centering
        \includegraphics[width=\textwidth]{convolve2d_fish.png}
        \caption{convolve2d output image: 307 $\times$ 375}
        \label{convolve2d_fish}
    \end{subfigure}
\end{figure}
\end{enumerate}

%%%%%%%%%%%%%%%%%%%%%%%%%%%%%%%%%%%

% Please leave the page break
\pagebreak
\paragraph{Q3:} Please review the 'Thinking in Frequency Part 1' lecture slides on aliasing, and think about how we might anti-alias with a low-pass filter. Related high-pass filters also exist. For (a--c), which kind of filter does the kernel represent? For (d--e), which filter produced the output images?

%%%%%%%%%%%%%%%%%%%%%%%%%%%%%%%%%%%
\emph{LaTeX:} To fill in boxes, replace `\textbackslash square' with `\textbackslash blacksquare' for your answer.

\begin{enumerate}[(a)]
\item
 $\begin{bmatrix}
    1 & 0 & -1 \\
    1 & 0 & -1 \\
    1 & 0 & -1 \\
 \end{bmatrix}$
\begin{tabular}[h]{ll}
$\blacksquare$ & High pass \\
$\square$ & Low pass \\
$\square$ & Neither \\
\end{tabular}

\item
 $\begin{bmatrix}
    \frac{1}{9} & \frac{1}{9} & \frac{1}{9} \\
    \frac{1}{9} & \frac{1}{9} & \frac{1}{9} \\
    \frac{1}{9} & \frac{1}{9} & \frac{1}{9}
 \end{bmatrix}$
\begin{tabular}[h]{ll}
$\square$ & High pass \\
$\blacksquare$ & Low pass \\
$\square$ & Neither \\
\end{tabular}

\item
$\begin{bmatrix}
    -\frac{1}{9} & -\frac{1}{9} & -\frac{1}{9} \\
    -\frac{1}{9} & \frac{8}{9} & -\frac{1}{9} \\
    -\frac{1}{9} & -\frac{1}{9} & -\frac{1}{9}
  \end{bmatrix}$
\begin{tabular}[h]{ll}
$\blacksquare$ & High pass \\
$\square$ & Low pass \\
$\square$ & Neither \\
\end{tabular}

\item
Input image:\\
\raisebox{\baselineskip-\height}{\includegraphics[width = 6cm]{q3img0.png}} \\
Output image 1:\\
\raisebox{\baselineskip-\height}{\includegraphics[width = 6cm]{q3img1.png}}
\begin{tabular}[h]{lc}
$\square$ & High pass \\
$\blacksquare$ & Low pass \\
\end{tabular}

\item
Output image 2:\\
\raisebox{\baselineskip-\height}{\includegraphics[width = 6cm]{q3img2.png}}
\begin{tabular}[h]{lc}
$\blacksquare$ & High pass \\
$\square$ & Low pass \\
\end{tabular}

\item
Which of the following statements are true? (Check all that apply).

\begin{tabular}[h]{ll}
$\blacksquare$ & High pass filter kernels will always contain at least one negative number \\
$\square$ & A Gaussian filter is an example of a low pass filter \\
$\square$ & A high pass filter is the basis for most smoothing methods \\
$\square$ & In a high pass filter, the center of the kernel must have the highest value \\
\end{tabular}

\end{enumerate}



%%%%%%%%%%%%%%%%%%%%%%%%%%%%%%%%%%%

% Please leave the pagebreak
\pagebreak
\paragraph{Q4:} 
\begin{enumerate}[(a)]
    \item 
    How does computation time vary with filter sizes from $3\times3$ to $15\times15$ (test odd and square sizes, i.e. $3\times3$, $5\times5$, $7\times7$, etc.), and with image sizes from approximately 0.25 to 8 megapixels? To find out fill out the below stencil to graph, for each of the above filter sizes, the correlation between an image size (x-axis) and time to convolve/correlate (y-axis) that image– each filter size should be its own line in the multi-line plot. 
    
    The stencil code imports the libraries you will need, but to understand how to use them you must look at the documentation.
    \begin{itemize}
    \item convolve/correlate - \href{https://docs.scipy.org/doc/scipy/reference/generated/scipy.ndimage.convolve.html}{$scipy.ndimage.convolve$} or \href{https://docs.scipy.org/doc/scipy/reference/generated/scipy.ndimage.correlate.html}{$scipy.ndimage.correlate$}
    \item rescale - \href{https://scikit-image.org/docs/dev/api/skimage.transform.html#skimage.transform.rescale}{$skimage.transform.rescale$}
    \item resize - \href{https://scikit-image.org/docs/dev/api/skimage.transform.html#skimage.transform.resize}{$skimage.transform.resize$}
    \item rescale vs resize – an example \href{http://scikit-image.org/docs/dev/auto_examples/transform/plot_rescale.html}{here}
    \end{itemize} 

    
    Add your graph as well as a brief description of what your graph demonstrates, below.

\emph{Note A megapixel is 1,048,576 ($2^{20}$) pixels (1024$\times$1024), or sometimes also 1,000,000 pixels (especially if you manufacture cameras). Megapixels is often shortened to MP or MPix.}

\emph{Image:} \href{RISDance.jpg}{RISDance.jpg} (in the .tex directory).

\begin{python}
import time 
import matplotlib.pyplot as plt
from skimage import io, img_as_float32
#use to rescale+resize image
from skimage.transform import rescale, resize
#use to convolve/correlate image
from scipy.ndimage import correlate
import numpy as np

#This reads in image and converts to a floating point format
# 1) TODO - replace PATH with the actual path to the 
#    downloaded RISDance.jpg image linked above
image = img_as_float32(io.imread('./RISDance.jpg'))

# 2) TODO - change the image size so it starts at 8MPix 
#    use one of the imported libraries
original_image = resize(image, (2160, 3840), anti_aliasing=True)

# 3) TODO - iterate through odd numbers from 3 to 15
#   (inclusive!!) these will represent your filter sizes
#   (3x3,5x5,7x7, etc.), for each filter size you will...
for kernel_size in range(3, 16):

    #because for each loop you are resizing your image, you 
    #want to start each loop w/the original image size
    shrinking_image = original_image 
    
    #these lists will hold the values you plot
    image_sizes = [] #x axis
    times = [] #y axis

    #while image size is bigger than .25MPx
    while(shrinking_image.size > 250000):
    
    	# 4) TODO - create your kernel. Your kernel can hold
    	#    any values, as the kernel values shouldn't 
    	#    affect computation time. The size of the kernel
    	#    must be kernel_size x kernel_size
        kernel = np.random.rand(kernel_size, kernel_size, 1)
        
        #5) TODO - reduce your image size. You can choose by 
        # what increments to reduce your image.
        shrinking_image = rescale(shrinking_image, 0.75)
        
        #gets the current time (in seconds)
        start = time.time() 
        
        # 6) TODO - use one of the imported libraries to do 
        # your correlation/convolution on the image. You can 
        # choose which operation to perform.
        convolve(shrinking_image, kernel)

        
        #gets the current time (in seconds)
        end = time.time() 
        
        #7) TODO - figure out what values to append, and 
        #   append them here
        image_sizes.append(shrinking_image.size)
        times.append(end-start)
    
    #each filter size will be plotted as a separate line, in 
    #a multi-line 2-dimensional graph
    plt.plot(image_sizes, times, label=str(kernel.size))

#plot
plt.xlabel('image size (pixels)')
plt.ylabel('operation time (seconds)')
plt.legend(title="filter sizes (pixels)")
plt.show()

\end{python}


\item
    Do the results match your expectation given the number of multiply and add operations in convolution?
\end{enumerate}

%%%%%%%%%%%%%%%%%%%%%%%%%%%%%%%%%%%
\pagebreak
\paragraph{A4:} Your answer here.
% Uncomment the stencil below and fill in your solution.

\begin{enumerate}[(a)]

\item Add your graph as well as a brief description of what your graph demonstrates. \\
The code is implemented as above. Below is the figure for excution time plot for convolution operation with different size of kernels and images. We can see that with the filter sizes and image sizes increaseing, the computation time increases as well.

\begin{figure*}[htbp]
    \centering
    \includegraphics[scale=0.5]{A4.png}
\end{figure*}
\item Do the results match your expectation given the number of multiply and add operations in convolution? \\
The number of multiply and add operations increases with the increase of kernel size and image size. As the excution time is proportional to the number of operations, it will also increase. Thus, the results match our expectation.

\end{enumerate}

%%%%%%%%%%%%%%%%%%%%%%%%%%%%%%%%%%%
\pagebreak 
\paragraph{Q5:} The coding portion of this project requires the use of the library \emph{numpy}, which provides fast computation with large multi-dimensional arrays and matrices. Here are some small exercises on basic numpy operations to get you started! Write \emph{one} numpy function to complete each of the following tasks.

Note that numpy is usually imported as
\begin{verbatim}
    import numpy as np
\end{verbatim}
at the top of the code file. You can then call numpy functions with \begin{verbatim}
    np.function_name()
\end{verbatim}
You are encouraged to test out your answers by creating your own python program, importing numpy and calling and printing the results of your solutions! Some numpy functions you might find useful are \href{https://numpy.org/doc/stable/reference/generated/numpy.squeeze.html}{np.squeeze}, \href{https://numpy.org/doc/stable/reference/generated/numpy.expand_dims.html}{np.expand\_dims}, \href{https://numpy.org/doc/stable/reference/generated/numpy.clip.html}{np.clip}, \href{https://numpy.org/doc/stable/reference/generated/numpy.pad.html}{np.pad}, and \href{https://numpy.org/doc/stable/reference/generated/numpy.zeros.html}{np.zeros}.

\begin{enumerate}[a.]
    \item Create an array of shape (320,640) filled with zeros.
    \item Given a variable called \emph{img} of shape (1, 1, 320, 640), create a new 2D array variable of shape (320, 640) from \emph{img}, i.e. remove all the 1-sized dimensions. 
    \item Conversely, say you have an array variable \emph{img} of shape (320, 640), create a new 2D array variable of shape (1, 320, 640) from \emph{img}, i.e. add a dimension. 
    \item Clip the image array, \emph{img}, so all its values lie within the range [-0.5, 0.5].
    \item With an RGB-image array, \emph{img}, of shape (320, 640, 3), retrieve the blue channel of the image while preserving all of \emph{img}'s dimensions and values. 
    \item With an RGB-image array, \emph{img}, of shape (320, 640, 3), retrieve the red and blue channels of the image. 
    \item Pad an \textbf{RGB} (remember, this means the image has 3 channels) image array, \emph{img}, with two zeros on either side of each row and 3 zeros values on either side of each column. Don't add zeros to the front or back (the color channel dimension).
\end{enumerate}


%%%%%%%%%%%%%%%%%%%%%%%%%%%%%%%%%%%
\pagebreak
\paragraph{A5:} Your answer here. Remember, write \emph{one} numpy function to complete each of the tasks.
%  Uncomment the stencil below and fill in your solution.

\begin{enumerate}[(a)]
\item \begin{verbatim}np.zeros((320, 640))\end{verbatim}
\item \begin{verbatim}np.squeeze(img, axis=(0,1))\end{verbatim}
\item \begin{verbatim}np.expand_dims(b, axis=0)\end{verbatim}
\item \begin{verbatim}np.clip(c, -0.5, 0.5)\end{verbatim}
\item \begin{verbatim}img[:, :, 2]\end{verbatim}
\item \begin{verbatim}np.stack((img[:, :, 0], img[:,:,2]), axis=2)\end{verbatim}
\item \begin{verbatim}np.pad(img, ((2, 2), (3, 3), (0, 0)), 'constant')\end{verbatim}
\end{enumerate}


%%%%%%%%%%%%%%%%%%%%%%%%%%%%%%%%%%%
% Please leave the pagebreak
\pagebreak
\paragraph{Q6:} 
In 1990, \emph{New York Times} photography critic Andy Grundberg \href{https://www.nytimes.com/1990/08/12/arts/photography-view-ask-it-no-questions-the-camera-can-lie.html}{stated that:}\\``In the future, readers of newspapers and
magazines will probably view news pictures more as
illustrations than as reportage, since they can no longer distinguish between a genuine image and one that has been manipulated.''

\begin{enumerate}[(a)]
    \item When is Grundberg's `future'? Why? (2--4 sentences)
    
    \item When is a news picture no longer genuine? Are any manipulations permissible, and if so, which ones? (2--4 sentences)
    
    \item If you worked for the \emph{New York Times} and were tasked with maintaining readers' trust in news pictures in Grundberg's future, what would you do? Consider everything that happens in the publication of a news picture. Describe your approach, and why it would work. (3--5 sentences)
    
    \begin{enumerate}[(i)]
    \item As per c), but instead you worked for the \emph{College Hill Independent}? \\(2--4 sentences)
    \end{enumerate}
    
    \item Include an cited example of a manipulated news picture, and identify the manipulation. What was the intent and impact of the manipulation? (2--4 sentences)
\end{enumerate}

\emph{Note:} These are open questions. We will grade for thought and justification. 

%%%%%%%%%%%%%%%%%%%%%%%%%%%%%%%%%%%

 \paragraph{A6:} Your answer here
%  Uncomment the stencil below and fill in your solution.
% \begin{enumerate}[(a)]
% \item
% \item 
% \item 
% \begin{enumerate}[(i)]
%     \item
% \end{enumerate}
% \item
% \end{enumerate}


%%%%%%%%%%%%%%%%%%%%%%%%%%%%%%%%%%%

%% any suggestions for more? 
\pagebreak
\section*{Feedback? (Optional)}
Please help us make the course better. If you have any feedback for this assignment, we'd love to hear it!


\pagebreak
\section*{A2: (b) code to obtain results}
\begin{python}
from skimage.color import rgb2gray
from skimage import io, color
import matplotlib
matplotlib.use("TkAgg")
import matplotlib.pyplot as plt
from scipy.signal import correlate2d, convolve2d
import numpy as np

testimg = io.imread('../data/fish.bmp')
testimg = rgb2gray(testimg)
# input image
testimg2 = testimg - np.mean(testimg)
plt.imsave('fish.png', testimg2, cmap=plt.cm.gray)
# filter 
f = testimg[125:175, 50:100]
f2 = f - np.mean(f)
plt.imsave('filter_fish.png', f2, cmap=plt.cm.gray)
# correlate output
I = correlate2d(testimg2, f2, 'same')
plt.imsave('correlate2d_fish.png', I, cmap=plt.cm.gray)
# convolve output
I2 = convolve2d(testimg2, f2, 'same')
plt.imsave('convolve2d_fish.png', I2, cmap=plt.cm.gray)
\end{python}

\end{document}
